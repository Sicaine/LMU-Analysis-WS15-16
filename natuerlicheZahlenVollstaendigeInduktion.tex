\chapter{Natürliche Zahlen, Vollständige Induktion}

\section{Die Axiome von Peano}
\paragraph{1. Axiom:}
1 ist eine natürliche Zahl\newline
1 $\in \mathbb{N}$

\paragraph{2. Axiom:}
Jede natürliche Zahl $n \in \mathbb{N}$ hat genau einen Nachfolger $n' \in \mathbb{N}$\newline
Bsp. \{1, 2, 3\} 1' = 2; 2' = 3; (3' = 1) $\rightarrow$ Da fehlt noch etwas

\paragraph{3. Axiom:}
1 ist niemals Nachfolger einer natürlichen Zahl\newline
Bsp. \{1, 2, 3\} 1'=2; 2'=3; (3' = 2) $\rightarrow$ Da fehlt noch etwas

\paragraph{4. Axiom:}
Falls $m'=n' \Rightarrow m=n$ (Jede natürliche Zahl ist Nachfolger von höchstens einer natürlichen Zahl)\newline
Bsp.: \{1, 2, 4, 5, ...\}  1'=2; 2'=4; ... Bezeichnungen sind irrelevant fuer den Mathematiker. Das sind die natürlichen Zahlen in einer komischen Sprache.\newline
Bsp.: \{1, 2, 3, ...\} $\cup$ \{a, b, c\} (a'=b; b'=c; c'=a) $\Rightarrow$ Da fehlt noch etwas\newline
\underline{Idee:} Wir nehmen die "kleinste" Menge, die alle Axiome 1-4 erfüllt.

\paragraph{5. Axiom:}
Sei X eine Menge mit:
\begin{enumerate}
\item 1 $\in$ X
\item Falls $n \in X \cap \mathbb{N} \Rightarrow n' \in X \newline \Rightarrow \mathbb{N} \subset X$ (echte Teilmenge $\subseteq$)
\end{enumerate}

% TODO: Einruecken
\subparagraph{Bemerkung:}
Durch das 5. Axiom ist die Eindeutigkeit von $\mathbb{N}$ sichergestellt. Heuristisch: Es gibt \underline{eine} kleinste Menge mit den Eigenschaften "1-4".

\paragraph{Bemerkung:}
Die natürlichen Zahlen sind durch obige Axiome mehr als nur eine Menge: Es gibt bereits eine gewisse Struktur. Wir haben über "Nachfolger" gesprochen. Diese Struktur wird Grundlage für weitere Dinge, insbesondere Addition und Multiplikation sein. Dazu später mehr.

\chapter{Vollständige Induktion}
Bei der V.I. handelt es sich um ein Beweisprinzip. Sie Ermöglicht den Beweis von Aussagen die für alle natürlichen Zahlen gelten sollen.\newline
Z.B.: $\sum_{n}^{j=1} j = \frac{n(n+1)}{2}$ gilt für alle $n \in \mathbb{N}$\newline
Es beruht auf dem Induktionsaxiom:\newline
Sei X die Menge aller Zahlen für die die Aussage gültig ist.
