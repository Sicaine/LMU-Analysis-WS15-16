\chapter{Natürliche Zahlen, Vollständige Induktion}

\section{Die Axiome von Peano}
\paragraph{1. Axiom:}
1 ist eine natürliche Zahl\newline
1 $\in \mathbb{N}$

\paragraph{2. Axiom:}
Jede natürliche Zahl $n \in \mathbb{N}$ hat genau einen Nachfolger $n' \in \mathbb{N}$\newline
Bsp. \{1, 2, 3\} 1' = 2; 2' = 3; (3' = 1) $\rightarrow$ Da fehlt noch etwas

\paragraph{3. Axiom:}
1 ist niemals Nachfolger einer natuerlichen Zahl\newline
Bsp. \{1, 2, 3\} 1'=2; 2'=3; (3' = 2) $\rightarrow$ Da fehlt noch etwas

\paragraph{4. Axiom:}
Falls $m'=n' \Rightarrow m=n$ (Jede nat. Zahl ist Nachfolger von hoechstens einer nat. Zahl)\newline
Bsp.: \{1, 2, 4, 5, ...\}  1'=2; 2'=4; ... Bezeichnungen sind irrelevant fuer den Mathematiker. Das sind die natuerlichen Zahlen in einer komischen Sprache.\newline
Bsp.: \{1, 2, 3, ...\} $\cup$ \{a, b, c\} (a'=b; b'=c; c'=a) $\Rightarrow$ Da fehlt noch etwas\newline
\underline{Idee:} Wir nehmen die "kleinste" Menge, die alle Axiome 1-4 erfuellt.

\paragraph{5. Axiom:}
Sei X eine Menge mit:
\begin{enumerate}
\item 1 $\in$ X
\item Falls $n \in X \cap \Rightarrow n' \in X \Rightarrow \mathbb{N} \subset X (echte Teilmenge \subseteq )$
\end{enumerate}

% TODO: Einruecken
\subparagraph{Bemerkung:}
Durch das 5. Axiom ist die Eindeutigkeit von $\mathbb{N}$ sichergestellt. Heuristisch: Es gibt \underline{eine} kleinste Menge mit Eig. "1-4"

\paragraph{Bemerkung:}
Die natuerlichen Zahlen sind durch obige axiome mehr als nur eine Menge: Es gibt bereits eine gewisse Struktur. Wir haben ueber "Nachfolger" gesprochen. Diese Struktur wird Grundlage fuer weitere Dinge, insbesondere Addition u. Multiplikation sein. Dazu spaeter mehr.

